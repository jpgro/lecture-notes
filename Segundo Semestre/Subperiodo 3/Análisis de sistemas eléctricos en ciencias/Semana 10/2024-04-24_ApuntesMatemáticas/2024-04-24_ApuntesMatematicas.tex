\documentclass[12pt, letterpaper]{report}
\usepackage{graphicx}
\usepackage{hyperref}
\usepackage{amssymb}
\usepackage{amsmath}
\usepackage{float}
\usepackage{mathtools}
\usepackage{enumitem}
\usepackage[margin=1in]{geometry}
\usepackage[figurename=Figura]{caption}
\title{Apuntes Módulo Matemáticas: Análisis de sistemas eléctricos en ciencias}
\author{Juan Pablo Guerrero Escudero}
\date{24 abril, 2024}
\begin{document}
\maketitle
\subsection*{Parametrización de curvas} 
Para parametrizar una curva, hacemos que x y y sean funciones de una tercera variable t, la cuál se le llama \textbf{parámetro} de la forma 
$x(t)$ y $y(t)$. Conforme $t$ varía, el resultado de las funciones cambian y se generan coordenadas $(x, y)$ que trazan una curva 
paramétrica. La manera de parametrizar no es única y pueden existir varias parametrizaciones de la misma curva. Para graficar una curva paramétrica en Python se hace lo siguiente: 
\begin{verbatim}
import numpy as np 
import matplotlib.pyplot as plt
all_x = []
all_y = []
while (t <= 2*np.pi): 
    x = t + np.sin(5*t)
    y = t + np.sin(6*t)
    t = t + 0.01
    all_x.append(x)
    all_y.append(y)
plt.plot(all_x, all_y)
plt.show()
\end{verbatim}
Después, si queremos encontrar la recta tangente a un punto sobre la curva, la pendiente o razón de cambio en un punto específico nos la 
da la regla de la cadena para funciones paramétricas: 
\begin{align}
\frac{dy}{dx} = \frac{\frac{dy}{dt}}{\frac{dx}{dt}} 
\end{align}si $\frac{dx}{dt} \neq 0$. La curva tiene una tangente horizontal cuando $\frac{dy}{dt} = 0$ y una tangente vertical cuando 
$\frac{dx}{dt} = 0$. Ahora, para calcular la concavidad se obtiene la segunda derivada de la función paramétrica de la siguiente forma: 
\begin{align}
\frac{d^2y}{dx^2} = \frac{d}{dx}(\frac{dy}{dx}) = \frac{\frac{d}{dt}(\frac{dy}{dx})}{\frac{dx}{dt}}
\end{align}La curva es cóncava hacia arriba cuanto la segunda derivada es positiva, y cóncava hacia abajo cuando ésta es negativa.    \\

\textbf{Área}: Para calcular el área de una ecuación paramétrica desde $t = \alpha$ hasta $t = \beta$, se calcula utilizando la siguiente regla de sustitución: 
\begin{align}
A = \int_{\alpha}^{\beta}ydx = \int_{\alpha}^{\beta} g(t)f'(t)dt
\end{align}
\textbf{Longitud de curva o arco}: Si tenemos ecuaciones paramétricas $x = f(t)$ y $y=g(t)$, y $\frac{dx}{dt} > 0$ en el intervalo $\alpha \leq t \leq \beta$, significa que 
la curva es recorrida una vez, de izquierda a derecha, cuanto $t$ incrementa de $\alpha$ a $\beta$. La fórmula de longitud de curva es: 
\begin{align}
L = \int_{\alpha}^{\beta} \sqrt{(\frac{dx}{dt})^2 + (\frac{dy}{dt})^2} dt
\end{align}Esto es porque en general, $L = \int ds$ y en éste caso $(ds)^2 = (dx)^2 + (dy)^2$. Igualmente, podemos extender ésta fórmula a 3 dimensiones:
\begin{align}
    L = \int_{\alpha}^{\beta} \sqrt{(\frac{dx}{dt})^2 + (\frac{dy}{dt})^2 + (\frac{dz}{dt})^2} dt
\end{align}En donde $z = f(t)$, y las derivadas de las 3 funciones paramétricas son continuas en $[\alpha, \beta]$. \\ 

Ahora, si se quiere encontrar la longitud de una recta que parte de $\vec{r}_0$ y termina en $\vec{r}_1$, se utiliza la siguiente fórmula: 
\begin{align}
    \vec{r} = (1-t)\vec{r}_0 + t \vec{r}_1
\end{align}Donde $0 \leq t \leq 1$. Es importante recordar que aquí se trabajan con vectores, por lo que para obtener las funciones $x(t)$ y $y(t)$ de la recta, 
se multiplican las coordenadas de $\vec{r}_0$ y se crea un nuevo vector de 3 coordenadas donde cada coordenada es el producto del vector 
y $(1-t)$, y lo mismo para $\vec{r}_1$. Al final, se suman ambos vectores resultantes, y se obtiene el vector $(x(t), y(t), h(t))$, dependiendo de las 
dimensiones con las que estemos trabajando. 
\subsection*{Campos escalares}
Un campo escalar es una región $D \subset \mathbb{R}^n$, en la que a cada punto $x \in D$ se le asigna un escalar $f(x)$, asociando a cada valor 
con un punto en el espacio. Algunos usos comunes son la distribución de la temperatura o la presión del gas en el espacio. \\ 

\textbf{Integrales de línea: } Si tenemos la curva suave $C$ definida por las curvas paramétricas $x(t)$ y $y(t)$, y $a \leq t \leq b$, entonces la 
integral de línea de la función $f$ a lo largo de C se define como 
\begin{align}
\int_c f(x, y)ds
\end{align}Ésta fórmula viene de particionar el intervalo $[a, b]$ en subintervalos dependientes del tiempo, evaluar la función en los extremos de esos intervalos, 
y crear subarcos de longitud $\delta S_i$, los cuáles se suman para obtener la longitud o valor de la integral de línea. Así, se toma el límite cuando $n \rightarrow \infty$ y 
se reemplaza con la notación de integral. 
Por lo tanto, para el cálculo de la integral de línea de f a lo largo de C se hace: 
\begin{align}
\int_c f(x, y)ds &= \int_c f(x,y) \sqrt{(\frac{dx}{dt})^2+(\frac{dy}{dt})^2 + (\frac{dz}{dt})^2}dt\\
\int_c f(x, y)ds &= \int_{a}^{b}f(x(t), y(t)) \sqrt{(\frac{dx}{dt})^2 + (\frac{dy}{dt})^2}dt
\end{align}Donde $ds$ es la longitud de curva se multiplica por la función de dos variables. El valor de la integral de línea no depende de la parametrización de la curva, 
siempre que ésta se recorra exactamente una vez cuando $t$ se incrementa de $a$ a $b$. \\ 

Ahora, si C es una curva suave por tramos, es decir, es una unión de una cantidad finita de curvas $C_1 + C_2 + ... C_n$, donde el punto inicial de $C_{i+1}$ es el punto final de $C_i$. Entonces, 
la integral de línea a lo largo de $f$ es las suma de las integrales de f a lo largo de cada curva C: 
\begin{align}
\int_C f(x, y)ds &= \int_{C_1}f(x, y)ds + \int C_2 f(x, y)ds + ... + \int_{C_n} f(x, y)ds
\end{align}

\end{document}