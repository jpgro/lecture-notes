\documentclass[12pt, letterpaper]{report}
\usepackage{graphicx}
\usepackage{hyperref}
\usepackage{amssymb}
\usepackage{amsmath}
\usepackage{float}
\usepackage{mathtools}
\usepackage{enumitem}
\usepackage[margin=1in]{geometry}
\usepackage[figurename=Figura]{caption}
\title{Examen Argumentativo Escrito Matemáticas y Computación: Análisis de sistemas eléctricos en ciencias}
\author{Juan Pablo Guerrero Escudero, A01706810}
\date{01 mayo, 2024}
\begin{document}
\maketitle
\textbf{Apegándome al Código de Ética de los Estudiantes del Tecnológico de Monterrey, me
comprometo a que mi actuación en este examen esté regida por la honestidad
académica.}
\subsection*{Matemáticas}
\begin{enumerate}
\item Argumenta detalladamente en base a un cálculo porque el siguiente es un campo vectorial
conservativo: $\vec{F} = (3x^2 + y, e^y + x)$. \\

\textbf{Solución}: Un campo vectorial es una función que asigna a cada punto $(x, y)$ un vector $\vec{F}(x, y)$ en el plano, y un campo vectorial 
se denomina conservativo si el valor de la integral de línea es independiente de la trayectoria, lo que significa que 
integrar sobre dos trayectorias diferentes nos da el mismo resultado, siempre que empiecen y terminen el mismo punto. La manera de verificar 
si el campo vectorial es conservativo, es si existe una función $f$ que sea el gradiente del campo vectorial, es decir $\vec{F} = \nabla f$. Esto
quiere decir que el gradiente de $f$ se puede expresar como $\nabla f = D_xf\hat{i} + D_yf\hat{j}$, que es la función vectorial 
que contiene las derivadar parciales de $f$. Además, por el Teorema de Clairaut, debe suceder que el orden de derivación 
de las derivadas parciales de segundo orden sea indistinto, es decir, $D_{xy}f = D_{yx}f$. Por lo tanto, si hay un campo vectorial 
de la forma $F = P\hat{i} + Q\hat{j}$, éste es conservativo si $D_yP = D_xQ$. \\ 

Por lo tanto, en la función dada, $P = 3x^2 + y$ y $Q = e^y+ x$. Obtenemos la derivada parcial con respecto a $y$ de $P$, y con respecto a 
$x$ de $Q$. y resulta $D_yP = 1$, y $D_xQ = 1$, y por lo tanto $D_yP = D_xQ$, y eso significa que el campo vectorial sí es conservativo, es decir, 
el valor de sus integrales de línea es independiente de la trayectoria tomada. 
\item Posteriormente encuentra la función $f$ tal que $\vec{F} = \nabla f$. \\ 

\textbf{Solución}: Para hacer esto, sabemos que $\vec{F} = \nabla f = 3x^2\hat{i} + e^y + x\hat{j}$, y como el vector gradiente 
contiene las derivadas parciales, $D_xf = 3x^2$, y $D_yf = e^y + x$. Para obtener la función original, podemos integrar una derivada 
parcial respecto a $x$ o $y$ dependiendo el caso, y utilizar la derivada parcial restante para obtener el valor de la constante 
resultante de la integral. \\
Primero, integramos $D_xf$ con respecto a $x$ para obtener $f(x, y)$:\\
\begin{align}
f(x, y) &= \int D_xf dx\\ 
f(x, y) &= \int (3x^2 + y) dx\\ 
f(x, y) &= \int x^3 + xy + C(y)
\end{align}
Para obtener la constante de integración $C(y)$ podemos derivar $f(x, y)$: 
\begin{align}
D_yf = x + C_y(y)
\intertext{Comparando con la derivada parcial definida al inicio, $D_yf = e^y + x$, tiene forzosamente que ser que $C_y(y) = e^y$}
\end{align}
Así, como obtuvimos $C_y(y)$, decir, la derivada de la constante de integración respecto a $y$, podemos integrar nuevamente 
con respecto a $y$ para obtener $C(y)$ y completar la función. Como $C_y(y) = e^y$, la integral con respecto a $y$ resulta la misma, es decir, 
$C(y) = e^y$. Y por lo tanto la función original de donde viene el gradiente o campo vectorial, resulta $f(x, y) = x^3 + xy + e^y$. \\ 

Para comprobar, si calculamos las derivadas parciales con respecto a $x$ y $y$ de $f(x, y)$, coincide con el vector gradiente o el campo vectorial: 
$D_xf = 3x^2 + y$, $D_yf = x + e^y$, y por lo tanto el procedimiento es correcto. 
\item Finalmente, encuentra $\int_{(10, 1)}^{(15, 2)}\vec{F} \cdot d \vec{r}$. \\

\textbf{Solución}: Para integrar lo anterior, podemos recordar que $W = \int_{a}^{b} \vec{F} \cdot d \vec{r} = f(b) - f(a)$, ya que por una parte 
el campo es conservativo, y eso significa que no importa la trayectoria, por lo que de acuerdo con el Teorema Fundamental 
del cálculo, la integral es lo mismo que la diferencia de la función original evaluada en $a$ y en $b$, y en éste caso $a = (10, 1)$, $b = (15, 2)$, y si hacemos lo anterior, 
$f((15, 2)) - f((10, 1)) = 3.4123x10^3 - 1.0127x10^3 = 2399.6$. Por lo tanto $W = 2399.6$. 
\end{enumerate}
\subsection*{Computación}
Instrucción: Para encontrar la raíz de una función no lineal podemos usar métodos numéricos como el
de Bisección o el de Newton-Raphson.\\ 
\begin{enumerate}
\item Escriba ambos algoritmos (pasos, no código): \\ 
El algoritmo deL método de Bisección trabaja con funciones continuas en un intervalo $[a, b]$. Primero, verifica si el producto de $f(a)\cdot f(b) < 0$. Si se cumple, 
se calcula un punto medio $m$ en el intervalo $[a, b]$, y se evalúa $f(m)$. Si ese valor es igual a cero, significa que encontramos la raíz buscada. En caso contrario, 
primero verificamos si $f(m)$ tiene un signo opuesto a $f(a)$ o $f(b)$, y en base a eso se redefine el intervalo $[a, b]$ como $[a, m]$ o $[m, b]$, dependiendo de en qué intervalo 
haya un cambio de signo. Así sucesivamente, se va acotando más el intervalo hasta que se alcance la precisión deseada, usualmente 
es un valor de tolerancia. \\ 

En segundo lugar, el algoritmo de Newton-Raphson hace lo siguiente: Ocupa una función, y un punto aproximado donde se encuentra
encontrar su raíz, es decir $x_0$. Después el siguiente valor de $x$ se calcula $x_{n+1} = x_n + \frac{f(x_n)}{f'(x_n)}$, que geométricamente significa 
la recta tangente que intersecta al eje $x$, y así sucesivamente se van calculando los valores de $x_{n+1}$ hasta que se pase un valor de tolerancia, usualmente 
calculado el valor de la función con respecto al eje $x$. 
\item ¿Describa cuáles son las ventajas y desventajas de cada método y qué factores influyen
en la elección del método más adecuado en cada caso?\\ 
Las ventajas del método de Bisección es que es mucho más seguro para garantizar la convergencia a la raíz en una función. Además, es facil de entender, y funciona bien 
para funciones continuas o derivables en todo el intervalo. La principal desventaja es que converge de manera lineal, por lo que 
ocupa más iteraciones para alcanzar una solución precisa. Además, éste método no puede usarse para encontrar raíces de funciones 
no lineales. \\ 
Por otro lado, el método de Newton-Raphson converge más rápido a la raíz que el método de bisección, además de que requiere 
menos iteraciones que el método de Bisección para encontrar la raíz. Y además, puede usarse en funciones no lineales. Sin embargo, su principal desventaja es que 
depende en gran medida de la aproximación inicial, ya que si es lejana de la solución, el método puede divergir. Además, 
requiere el cálculo de las derivadas de la función, que puede suceder que no todas las funciones tengan derivadas explícitas o que se pueden 
expresar analíticamente. \\ 

Los factores que influyen en la decisión de elección del método adecuado, es primero si la función es lineal o no lineal, si la función tiene derivadas calculables analíticamente, y si 
se prefiere simplicidad sobre eficiencia. En el caso de que se necesite una convergencia rápida, y se tiene una función continua y derivable, es mejor usar el método de 
Newton Raphson porque requiere menor poder computacional. Por otro lado, si la función no es derivable, y se prefiere un método más 
simple, es mejor usar el método de Bisección. 

\item Cuando NO se pueden aplicar los métodos? Ambos métodos no se pueden aplicar si se tienen funciones discontinuas, ya que ambos 
asumen que la función es continua en un intervalo. Además, el Método de Newton-Raphson funciona con funciones derivables, entonces si no se puede 
derivar la función, no se puede usar éste método. Por último, si la función tiene múltiples raíces, ambos métodos pueden converger 
a la solución incorrecta. 
\end{enumerate}
\end{document}