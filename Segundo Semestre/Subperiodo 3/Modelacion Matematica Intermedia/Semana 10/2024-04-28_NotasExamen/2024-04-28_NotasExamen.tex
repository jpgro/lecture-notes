\documentclass[12pt, letterpaper]{report}
\usepackage{graphicx}
\usepackage{hyperref}
\usepackage{amssymb}
\usepackage{amsmath}
\usepackage{float}
\usepackage{mathtools}
\usepackage{enumitem}
\usepackage[margin=1in]{geometry}
\usepackage[figurename=Figura]{caption}
\title{Notas para examen}
\author{Juan Pablo Guerrero Escudero}
\date{28 abr, 2024}
\begin{document}
\maketitle
\subsection*{Derivadas parciales}
Las derivadas parciales de una función en un punto $(x_0,y_0)$ representan las pendientes de las rectas 
tangentes a las trazas en los planos $y = y_0$ y $x = x_0$. Geométricamente, representan 
lo mismo que la derivada univariada, considerando que las otras variables independientes permanecen constantes. \\
Para el cálculo de derivadas parciales, todas las reglas de derivación aplican, solo teniendo en cuenta que una de las variables
fija o constante, y así se trata como una función de una sola variable. \\ 

Las derivadas de primer y orden superior nos dan información sobre la función, sobre donde es creciente/decreciente, los puntos 
críticos y su naturaleza, y dónde es cóncava (hacia arriba o hacia abajo). \\

\textbf{Derivadas de orden superior}: Para una función de dos variables existen 4 derivadas de orden superior, 
$D_{xx}f$, $D_{xy}f$, $D_{yx}f$ y $D_{yy}f$. A las derivadas parciales de segundo orden $D_{xy}f$ y $D_{yx}f$ se le llaman 
mixtas o cruzadas. Ojo, para la notación de Jacobi, el orden de derivación se lee de derecha a izquierda, y en la 
notación de Euler se lee de izquierda a derecha. \\ 

Si la función es continua en su dominio, satisface el Teorema de Clairaut o de Schwarz, que dice que si las funciones 
$D_{xy}f(x_0, y_0)$ y $D_{yx}f(x_0, y_0)$ son continuas en un disco $D$ que contiene al punto $(x_0, y_0)$, entonces 
$D_{xy}f(x_0, y_0) = D_{yx}f(x_0, y_0)$. Es decir, el orden de derivación es indistinto. Éste teorema aplica no solo para derivadas 
de segundo orden, sino para derivadas de orden superior también. 

\subsection*{Planos tangentes}
Un plano tangente es un plano que contiene a las rectas tangentes a las trazas verticales de una superficie. Hay dos ecuaciones para obtener los planos tangentes. 
Si la superficie está definida por una función $z = f(x, y)$: 
\begin{align}
z-z_0 = D_xf(x_0, y_0)(x-x_0) + D_yf(x_0, y_0)(y-y_0)
\end{align}
En cambio si la superficie está definida por una ecuación $f(x, y, z) = c$: 
\begin{align}
0 = D_xf(x_0, y_0, z_0)(x-x_0) + D_yf(x_0, y_0, z_0)(y-y_0) + D_zf(x_0, y_0, z_0)
\end{align}
Para escribir las ecuaciones del plano tangente, a veces conviene reescribir a las funciones despejando para $z$ con el fin de convertirla en 
una función, y usar la fórmula del plano tangente a una función para obtenerlo. \\ 

\textbf{Aproximaciones lineales}: Una aproximación lineal o linearización a una función de dos variables en $(x_0, y_0)$ está definida por: 
\begin{align}
L(x, y) = f(x_0, y_0) + D_xf(x_0, y_0)(x-x_0) + D_yf(x_0, y_0)(y-y_0)
\end{align}Se puede ver la linearización como lo mismo que la ecuación del plano tangente a la superficie 
definida por $z = f(x, y)$ en $(x_0, y_0)$, donde $z_0 = f(x_0, y_0)$. \\ 

\textbf{Diferenciales}: Si recordamos que la recta tangente sirve como aproximación lineal, entonces cuando hay un cambio en $x$ de tamaño 
$\Delta  x$ existe el cambio que experimenta la función $y = f(x)$ denotado por $\Delta y$, y el que experimenta la 
recta tangente denotado por $dy$. En dos variables, se definen los diferenciales $dx$ y $dy$ como las variables independientes, y el diferencial total $dz$ como: 
\begin{align}
dz = D_xf(x, y)dx + D_y(x, y)dy
\end{align}

\subsection*{Derivadas direccionales y vector gradiente}
Las derivadas parciales nos dicen el cambio cuando solamente una variable está fija. Es decir, lo podemos pensar como solo movimiento en $x$ o 
solamente movimiento en $y$. Las derivadas direccionales en cambio nos dicen el cambio que tiene la función en cierta dirección, dada por un vector 
$\vec{u}$. Éstas nos permiten conocer el cambio máximo o mínimo de una función.\\

La derivada direccional $D_uf$ representa la taza o razón de cambio de una función $f$ en cualquier dirección $\vec{u}$, donde $\vec{u} = <a, b>$ un vector unitario. La 
forma operacional de la derivada direccional es la siguiente, teniendo un vector unitario como $\vec{u}$: 
\begin{align}
D_uf(x_0, y_0) = D_xf(x_0, y_0)a + D_yf(x_0, y_0)b
\end{align}
En cambio, si no se da el vector pero un ángulo $\theta$ de la dirección del vector, el vector unitario se vuelve $\vec{u} = <\cos(\theta), \sin(\theta)>$. 
\textbf{Ojo}: Si se nos da un vector, y no es unitario, podemos hacerlo unitario calculando su norma o magnitud, y dividiendo $a$ y $b$ entre la norma, de tal manera 
que el vector se convierta en unitario. \\ 
El resultado de una derivada direccional se puede interpretar como que la función $f$ crece o decrece $x$ unidades por cada 
unidad que nos desplazamos en dirección del vector $\vec{u}$, o bien, desplazándonos en dirección del vector unitario. \\ 

Tips: Siempre busca simplificar el vector unitario, y usar el sistema correcto de unidades. De ser posible, traslada el 
resultado a una forma exacta por medio del álgebra, en vez de meter todo en la calculadora. \\ 

Otra forma de expresar la derivada direccional es como el producto punto entre el gradiente de f, $\nabla f = <D_xf(x_0, y_0), D_yf(x_0, y_0)>$ y $u = <a, b>$.: 
\begin{align}
D_uf(x_0, y_0) &= \nabla f(x_0, y_0) \cdot \vec{u}\\
D_uf(x_0, y_0) &= ||\nabla f(x_0, y_0)|| \cdot ||\vec{u}|| \cos(\theta)\\
\intertext{Y como el vector $\vec{u}$ es unitario:}
D_uf(x_0, y_0) &= ||\nabla f (x_0, y_0) || \cdot \cos(\theta)
\end{align}
En el proceso anterior, donde $\theta$ es el ángulo entre los vectores $\nabla f$ y $\vec{u}$. Nota, no es lo mismo que la 
dirección del vector unitario, no confundir. Simplificamos porque la magnitud de $\vec{u}$ es 1, y por lo tanto notamos que 
los valores de la direccional en un punto dependen de $\theta$ ya que $\||\nabla f||$ es constante en $(x_0, y_0)$. Entonces, podemos deducir 3 cosas: 
\begin{enumerate}
\item $D_uf(x_0, y_0)$ se maximiza cuando $\cos(\theta) = 1$, o $\theta = 0$. 
\item $D_uf(x_0, y_0)$ se minimiza cuando $\cos(\theta) = -1$, o $\theta = \pi$. 
\item $D_uf(x_0, y_0)$ se anula cuando $\cos(\theta) = 0$, o $\theta = \frac{\pi}{2}$. 
\end{enumerate}
Por lo tanto, lo anterior quiere decir que si $\nabla f$ o vector gradiente es diferente a cero, la dirección de máximo crecimiento es la misma 
que el vector gradiente evaluado en un punto, y el valor máximo de la direccional es $||\nabla f(x_0, y_0)||$. Además, la dirección de máximo decrecimiento es 
la dirección opuesta al gradiente, y el valor mínimo de la direccional $-||\nabla f(x_0, y_0)||$. Por último, si $\nabla f(x_0, y_0) \neq 0$, en las direcciones ortogonales 
o perpendiculares al vector gradiente no hay cambio. \\

Para obtener la dirección del vector gradiente teniendo sus coordenadas, se hace $\theta = \tan^-1(\frac{y}{x})$, donde $x$ y $y$ son las componentes del vector 
gradiente.\\

Una última cosa es que si el vector gradiente es igual a cero, entonces la derivada direccional es igual a cero en cualquier dirección. \\ 

En realidad el gradiente es un campo vectorial, ya que a cada punto en el plano o espacio asigna un vector que da la máxima razón de cambio en ese punto. 

\subsection*{Regla de la cadena para funciones de varias variables}
Ver presentación en Canva 
\subsection*{Integrales dobles}
En integrales dobles, se tiene la noción de que el integrando de éstas se forma haciendo "curva de arriba" menos curva de abajo, o curva a la derecha menos curva 
a la izquierda, y la variable que se mueve sobre el intervalo tiene límites constantes. \\
Una región D en el plano es de: 
\begin{enumerate}
\item Tipo 1: se ubica entre dos gráficas de dos funciones continuas de X. Esto es, D queda definida por 
$(x, y :  a \leq x \leq b, g_1(x) \leq y \leq g_2(x))$. Es decir, $x$ se mueve entre constantes, y $y$ se ajusta a la variable 
x por medio de alguna función. 
\item Tipo 2: se ubica entre las gráficas de dos funciones continuas de Y, esto es, D queda definida por: 
$(x, y : h_1(x) \leq x \leq h_2(y), c \leq y \leq d)$. Es decir, $y$ se mueve entre constantes, y $x$ se ajusta a la variable $y$ por medio 
de alguna función \\ 

Para verificar si es tipo 1 (Curva de arriba - curva de abajo) o tipo 2 (curva de derecha - curva de izquierda), se trazan "flechas"
desde una a curva a otra, y buscamos que sean continuas las flechas, es decir, que se todas las flechas empiezen en la misma curva y terminen 
en otra curva. 
\end{enumerate}
\textbf{Integrales dobles: } Si la región de integración $D$ es de Tipo 1, entonces la integral doble resutla: 
\begin{align}
{\int\int}_D f(x, y) dA = \int_{a}^{b} \int_{g_1(x)}^{g_2(x)} f(x, y) dydx
\end{align}
Y si la región de integración es tipo 2: 
\begin{align}
    \int \int_Df(x, y) dA = \int_{c}^{d} \int_{h_1(x)}^{h_2(x)} f(x, y) dxdy
\end{align}
En cualquier caso, el orden de integración nos debe dar el mismo valor del volumen, y de ahí viene el Teorema de Fubini, 
que dice que si $f$ es una región continua en $D$, entonces: 
\begin{align}
    \int \int_Df(x, y) dA = \int_{a}^{b} \int_{g_1(x)}^{g_2(x)} f(x, y) dydx = \int_{c}^{d} \int_{h_1(x)}^{h_2(x)} f(x, y) dxdy
\end{align}
Y por lo tanto podemos decidir un orden de integración sobre otro para facilitar la evaluación. Ojo: en la integral más externa 
nunca van variables, solo constantes, en ambos casos (Tipo I y Tipo II). 

\subsection*{Área superficial}
El área de una superficie $S$ definida por $z = f(x, y)$ en una región $D$ está dada por la integral doble
\begin{align}
A_s = \int \int_D \sqrt{(D_xf)^2 + (D_yf)^2 + 1} dA
\end{align}. Ésta tiene gran similitud con la fórmula de longitud de arco en funciones univariadas: $L = \int_{a}^{b} \sqrt{1 + f'(x)^2} dx$. Aquí aplican 
igualmente los conceptos para determinar la región de integración en el apartado de integrales dobles. Como las integrales 
dobles resultan muchas veces muy complejas de resolver, se usa software como Wolfram o Matlab para resolverlas. 
\end{document}