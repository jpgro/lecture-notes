\documentclass[12pt, letterpaper]{report}
\usepackage{graphicx}
\usepackage{hyperref}
\usepackage{amssymb}
\usepackage{amsmath}
\usepackage{float}
\usepackage{mathtools}
\usepackage{enumitem}
\usepackage[margin=1in]{geometry}
\usepackage[figurename=Figura]{caption}
\title{Ecuaciones Diferenciales y Planos Tangentes}
\author{Juan Pablo Guerrero Escudero}
\date{04 abril, 2024}
\begin{document}
\maketitle
\subsection*{Ecuaciones Diferenciales}
Las ecuaciones diferenciales son ecuaciones que contienen derivadas dentro de ellas. Es decir, son funciones 
que relacionan una función con sus derivadas. Éstas nos ayudan, entre otras cosas, a describir cómo cambian las 
cosas con el tiempo. Se pueden clasificar de dos tipos: 
\begin{enumerate}
\item Ecuaciones diferenciales ordinarias: Son aquellas que contienen derivadas no parciales. 
\item Ecuaciones diferenciales no ordinarias: Son aquellas que contienen derivadas parciales. 
\end{enumerate}
En clase, se realizó el ejercicio de verificar si una función satisface o no la Ecuación Diferencial Parcial (EDP) de 
Laplace en la forma ($D_{xx}U + D_{yy}U = 0$). Si satisface lo anterior, se dice que $U$ es armónica. A continuación se 
muestra el procedimiento de algunos ejercicios: \\ 
\begin{enumerate}
\item $u(x_y) = x^3 + 3xy^2$: 
\begin{align*}
    D_{xx}u =& 6x \\
    D_{yy}u =& 6x \\
    6x + 6x \neq& 0
\end{align*} Y por lo tanto no cumple con la EDP de Laplace. 
\item $u(x, y) = Ln(\sqrt{x^2 + y^2})$: 
\begin{align*}
D_{xx}u =& \frac{y^2 - x^2}{(x^2 + y^2)^2}\\
D_{yy}u =& \frac{x^2 + y^2}{(x^2 + y^2)^2} \\
\frac{y^2 - x^2}{(x^2 + y^2)^2} - \frac{x^2 + y^2}{(x^2 + y^2)^2} =& \frac{0}{(x^2 + y^2)^2}
\end{align*} Y por lo tanto sí cumple con la EDP de Laplace. 
\item $u(x, y) = e^{-x}cos(y) -e^{-y}cos(x)$
\begin{align*}
D_{xx}u =& e^{-y}cos(x) + e^{-x}cos(y)\\
D_{yy}u =& -e^{-y}cos(x) - e^{-x}cos(y)
D_{xx}u + D_{yy}u =& 0
\end{align*} Por lo tanto, sí se cumple la ecuación diferencial de Laplace. 
\end{enumerate}
Observación: Hay muchas ecuaciónes diferenciales en el mundo de las matemáticas, y la mayoría de ellas son muy difíciles de 
resolver analíticamente, por lo que muchas veces se resuelven por métodos numéricos o de aproximación. 
\subsection*{Planos Tangentes}
Los planos tangentes son las aproximaciones lineales a las superficies. También, se puede ver como el 
plano que contiene a las rectas tangentes a las trazas verticales de una superficie. Entonces, 
Se le puede dar a $D_{x}f(x_0, y_0)$ la interpretación de pendiente de la recta tangente a la traza en $y = y_0$, y 
análogamente, $D_yf(x_0, y_0)$ la pendiente de la recta tangente a la traza en $x = x_0$. \\ 

Si $z_0 = f(x_0, y_0)$, entonces las rectas tangentes son: 
\begin{align}
RT_x = z &= f(x_0, y_0) + D_xf(x_0, y_0)[x-x_0] \\
RT_y = z &= f(x_0, y_0) + D_yf(x_0, y_0)[y-y_0]
\end{align} Entonces, la ecuación del plano tangente es: 
\begin{align}  
z = f(x_0, y_0) + D_xf(x_0, y_0)[x-x_0] + D_yf{x_0, y_0}[y-y_0]
\end{align} En donde z siempre contiene a ambas rectas tangentes, se puede ver como el punto inicial $f(x_0, y_0)$ más 
la Razón de cambio de $x$, ($D_xf(x_0, y_0)[x-x_0]$) más la razón de cambio en $y$ ($D_yf(x_0, y_0)[y-y_0]$). \\

La ecuación del plano tangente también se puede ver como una aproximación lineal, que dice que si te acercas lo suficiente, los valores 
de la recta tangente se parecen a los de la superficie. Y por lo tanto, la ecuación de la aprox. lineal de una función es: 
\begin{align}
L(x, y) = f(x_0, y_0) + D_xf(x_0, y_0)[x-x_0] + D_yf(x_0, y_0)[y-y_0]
\end{align} A continuación, se muestra el ejemplo del cálculo de la ecuación del plano tangente: \\
\textbf{Ejemplo 1}: Determina el plano tangente a $x^2 + y^2 + z - 9 = 0$ en el punto $(1, 2, 4)$1.\\ 
\begin{enumerate}
\item En primer lugar, se puede reescribir la ecuación en términos de $z$ como $z = 9 - x^2 + y^2$. Ésto con el fin de hacerla en la forma de la ecuación del 
plano tangente $z = f(x, y)$. 
\item En segundo lugar, se determinan las derivadas parciales $D_xZ$ y $D_yZ$.
\begin{align*}
D_xZ =& -2x\\
D_yZ =& -2y
\end{align*}
\item Después, se escribe la aproximación lineal de la función, o el plano tangente (es lo mismo). 
\begin{align*}
L(1, 2) =& f(1, 2) - 2(1)[x - 1] -2(2)[y - 2]\\
L(1, 2) =& 4 -2x + 2 -4y + 8\\
L(1, 2) =& -2x -4y + 14\\ 
z =& -2x -4y + 14
\end{align*} La cuál es la ecuación del plano tangente a la ecuación. 
\end{enumerate}

\textbf{Ejercicio de práctica}: La ecuación del plano tangente a una superficie es: 
$P_{tangente} = D_xf(x_0, y_0, z_0)(x-x_0) + D_yf(x_0, y_0, z_0)(y-y_0) + D_zf(x_0, y_0, z_0)(z-z_0) = 0$. \\ 

\begin{enumerate}
    \item $z = f(x, y, z) = \frac{x^2}{a^2} - \frac{y^2}{b^2} + h$. 
    En éste caso se usa la fórmula de plano tangente a una función $f(x, y)$: 
     
\end{enumerate}
\end{document}