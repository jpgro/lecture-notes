\documentclass[12pt, letterpaper]{report}
\usepackage{multicol}
\usepackage{graphicx}
\usepackage{hyperref}
\usepackage{amssymb}
\usepackage{amsmath}
\usepackage{float}
\usepackage{mathtools}
\usepackage{enumitem}
\usepackage[margin=1in]{geometry}
\usepackage[figurename=Figura]{caption}
\title{Formulario electromagnetismo}
\author{Juan Pablo Guerrero Escudero}
\date{19 mayo, 2024}
\begin{document}
\maketitle
\begin{multicols*}{3}
\begin{enumerate}
    % Lorentz Force
    \item $\vec{F_B} = q\vec{v} \times \vec{B}$.
    \item $F_B = |q|v_\perp B = |q| v B_\perp = |q| v B \sin(\phi)$.
    % Force on a Current-Carrying Conductor
    \item $F_B = I \int_{P_i}^{P_f} d\vec{L} \times \vec{B}$.
    \item $F_B = I\vec{L} \times \vec{B} = I l B_\perp = I l B \sin(\phi)$.
    \item $\vec{L} = P_f - P_i$.
    % Electric Force
    \item $F_E = q\vec{E}$.
    % Total Force on a Moving Charge
    \item $\vec{F_T} = q(\vec{E} + \vec{v} \times \vec{B})$.
    % Radius of Circular Motion
    \item $R = \frac{mv}{|q|B}$.
    % Frequency and Angular Frequency
    \item $f = \frac{\omega}{2\pi} = \frac{qB}{2\pi m}$.
    \item $\omega = \frac{|q|B}{m}$.
    % Velocity
    \item $v = \frac{E}{B} = \frac{d}{t}$.
    % Electric Potential Difference
    \item $DV = -\vec{E} \cdot d$.
    % Torque and Magnetic Moment
    \item $T_r = I\vec{A} \times \vec{B} = IA B \sin(\phi)$.
    \item $\mu = I\vec{A} = nI\vec{A}$.
    % Biot-Savart Law
    \item $\vec{B} = \frac{\mu_0}{4\pi}\left(\frac{q \vec{v} \times \hat{r}}{r^2}\right)$.
    \item $B = \frac{\mu_0}{4\pi}\left(\frac{|q|v \sin(\phi)}{r^2}\right)$.
    % Magnetic Field of a Current Element
    \item $d\vec{B} = \frac{\mu_0 I}{4\pi}\left(\frac{d\vec{l} \times \hat{r}}{r^2}\right)$.
    \item $dB = \frac{\mu_0 I}{4\pi}\left(\frac{dl \sin(\phi)}{r^2}\right)$.
    % Magnetic Field of a Straight Conductor
    \item $\vec{B} = \frac{\mu_0 I}{4\pi a}(\cos{\theta_1} - \cos{\theta_2}) \hat{k}$.
    \item $\vec{B} = \frac{\mu_0 I}{2\pi r}$.
    \item $\vec{B} = \frac{\mu_0 I(\theta)}{4\pi r}$, $\theta \, \text{rad}$.
    \item $\vec{B} = \frac{\mu_0 I a^2}{2(x^2 + a^2)^{\frac{3}{2}}}$, $a = r$, $x = \text{dist}$.
    \item $B = \frac{\mu_0 I}{2r}$, $\theta = 2\pi$.
    \item $\vec{B} = \frac{\mu_0 I}{2r}(\cos{\theta}) \hat{i}$, $\theta \, \text{rad}$.
    % Magnetic Flux and Ampere's Law
    \item $\phi_B = \int_S \vec{B} \cdot d\vec{A} = 0$, SS.
    \item $C = \int_c \vec{B} \cdot d\vec{l} = \mu_0 I_{enc}$.
    \item $C = \int_c \vec{B} \cdot d\vec{l} = \frac{\mu_0 Ir}{2\pi R^2}$, $r < R$.
    \item $C = \int_c \vec{B} \cdot d\vec{l} = \mu_0 n I l$, $B = \mu_0 n I$.
\end{enumerate}
\end{multicols*}
\end{document}
